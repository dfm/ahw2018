\documentclass[letterpaper,12pt,preprint]{hack_aastex}

% Based on the `hogg_nasa` stylesheet by David W. Hogg (NYU)

% Colors.
\usepackage{amsmath}
\usepackage{color}
\usepackage[pagebackref=false]{hyperref}
\definecolor{linkcolor}{rgb}{0,0,0.25}
\hypersetup{colorlinks=true,linkcolor=linkcolor,citecolor=linkcolor,
            filecolor=linkcolor,urlcolor=linkcolor}

\usepackage[para]{footmisc}
\urlstyle{sm}

\usepackage{epsfig}
\usepackage{graphicx}
\setlength{\headheight}{2ex}
\setlength{\headsep}{2ex}

\usepackage{multicol}

\newcommand{\dd}{\,\mathrm{d}}
\newcommand{\bvec}[1]{{\ensuremath{{\boldsymbol{#1}}}}}
\newcommand{\transpose}[1]{{#1}^{\!\mathsf{T}}}
\newcommand{\inverse}[1]{{#1}^{-1}}


%----- exact 1-in margins
% NB: headheight and headsep MUST exist and be set
\setlength{\textwidth}{6.5in}
\setlength{\textheight}{9in}
\addtolength{\textheight}{-1.0\headheight}
\addtolength{\textheight}{-1.0\headsep}
\setlength{\topmargin}{-2em}
\addtolength{\textheight}{-1.0\topmargin}
\setlength{\oddsidemargin}{0.0in}
\setlength{\evensidemargin}{0.0in}

%----- typeset certain kinds of words
\newcommand{\observatory}[1]{\textsl{#1}}
\newcommand{\package}[1]{\textsf{#1}}
\newcommand{\project}[1]{\package{#1}}
\newcommand{\an}{\package{Astrometry.net}}
\newcommand{\NASA}{\observatory{NASA}}
\newcommand{\JWST}{\observatory{JWST}}
\newcommand{\Kepler}{\observatory{Kepler}}
\newcommand{\kepler}{\Kepler}
\newcommand{\Ktwo}{\observatory{K2}}
\newcommand{\ktwo}{\Ktwo}
\newcommand{\PLATO}{\observatory{PLATO}}
\newcommand{\MAST}{\observatory{MAST}}
\newcommand{\EA}{\observatory{Exoplanet Archive}}
\newcommand{\TESS}{\observatory{TESS}}
\newcommand{\galex}{\observatory{GALEX}}
\newcommand{\Spitzer}{\observatory{Spitzer}}
\newcommand{\gaia}{\observatory{Gaia}}
\newcommand{\Gaia}{\gaia}
\newcommand{\lsst}{\observatory{LSST}}
\newcommand{\sdss}{\observatory{SDSS}}
\newcommand{\latin}[1]{\textit{#1}}
\newcommand{\eg}{\latin{e.g.}}
\newcommand{\etal}{\latin{et~al.}}
\newcommand{\etc}{\latin{etc.}}
\newcommand{\ie}{\latin{i.e.}}
\newcommand{\vs}{\latin{vs.}}

\newcommand{\doi}[2]{\href{http://dx.doi.org/#1}{\emph{#2}}}
\newcommand{\ads}[2]{\href{http://adsabs.harvard.edu/abs/#1}{\emph{#2}}}
\newcommand{\arxiv}[1]{\href{http://arxiv.org/abs/#1}{arXiv:#1}}

%----- stuff for this proposal in particular.
\newcommand{\kplr}{\package{kplr}}
\newcommand{\PLM}{\package{PLM}}
\newcommand{\OWL}{\package{OWL}}
\newcommand{\George}{\package{George}}
\newcommand{\kpsf}{\package{kpsf}}
\newcommand{\emcee}{\package{emcee}}
\newcommand{\GitHub}{\package{GitHub}}
\newcommand{\KIC}{\textsl{KIC}}
\newcommand{\KOI}{\textsl{KOI}}

%----- typeset journals
% \newcommand{\aj}{Astron.\,J.}
% \newcommand{\apj}{Astrophys.\,J.}
% \newcommand{\apjl}{Astrophys.\,J.\,Lett.}
% \newcommand{\apjs}{Astrophys.\,J.\,Supp.\,Ser.}
% \newcommand{\mnras}{Mon.\,Not.\,Roy.\,Ast.\,Soc.}
% \newcommand{\aap}{Astron.\,\&~Astrophys.}

%----- Tighten up paragraphs and lists
\setlength{\parskip}{0.0ex}
\setlength{\parindent}{0.2in}
\renewenvironment{itemize}{\begin{list}{$\bullet$}{%
  \setlength{\topsep}{0.0ex}%
  \setlength{\parsep}{0.0ex}%
  \setlength{\partopsep}{0.0ex}%
  \setlength{\itemsep}{0.0ex}%
  \setlength{\leftmargin}{1.5\parindent}}}{\end{list}}
\newcounter{actr}
\renewenvironment{enumerate}{\begin{list}{\arabic{actr}.}{%
  \usecounter{actr}%
  \setlength{\topsep}{0.0ex}%
  \setlength{\parsep}{0.0ex}%
  \setlength{\partopsep}{0.0ex}%
  \setlength{\itemsep}{0.0ex}%
  \setlength{\leftmargin}{1.5\parindent}}}{\end{list}}

%----- mess with paragraph spacing!
\makeatletter
\renewcommand\paragraph{\@startsection{paragraph}{4}{\z@}%
                                    {1ex}%
                                    {-1em}%
                                    {\normalfont\normalsize\bfseries}}
\makeatother

%----- Special Hogg list for references
  \newcommand{\hogglist}{%
    \rightmargin=0in
    \leftmargin=0.25in
    \topsep=0ex
    \partopsep=0pt
    \itemsep=0ex
    \parsep=0pt
    \itemindent=-1.0\leftmargin
    \listparindent=\leftmargin
    \settowidth{\labelsep}{~}
    \usecounter{enumi}
  }

%----- side-to-side figure macro
%------- make numbers add up to 94%
 \newlength{\figurewidth}
 \newlength{\captionwidth}
 \newcommand{\ssfigure}[3]{%
   \setlength{\figurewidth}{#2\textwidth}
   \setlength{\captionwidth}{\textwidth}
   \addtolength{\captionwidth}{-\figurewidth}
   \addtolength{\captionwidth}{-0.02\figurewidth}
   \begin{figure}[htb]%
   \begin{tabular}{cc}%
     \begin{minipage}[c]{\figurewidth}%
       \resizebox{\figurewidth}{!}{\includegraphics{#1}}%
     \end{minipage} &%
     \begin{minipage}[c]{\captionwidth}%
       \textsf{\caption[]{\footnotesize {#3}}}%
     \end{minipage}%
   \end{tabular}%
   \end{figure}}

 \newcommand{\ssfiguretwo}[4]{%
   \setlength{\figurewidth}{#3\textwidth}
   \setlength{\captionwidth}{\textwidth}
   \addtolength{\captionwidth}{-\figurewidth}
   \addtolength{\captionwidth}{-0.02\figurewidth}
   \begin{figure}[htb]%
   \begin{tabular}{cc}%
     \begin{minipage}[c]{\figurewidth}%
       \resizebox{\figurewidth}{!}{\includegraphics{#1}}\\
       \resizebox{\figurewidth}{!}{\includegraphics{#2}}%
     \end{minipage} &%
     \begin{minipage}[c]{\captionwidth}%
       \textsf{\caption[]{\footnotesize {#4}}}%
     \end{minipage}%
   \end{tabular}%
   \end{figure}}

%----- top-bottom figure macro
 \newlength{\figureheight}
 \setlength{\figureheight}{0.75\textheight}
 \newcommand{\tbfigure}[2]{%
   \begin{figure}[htp]%
   \resizebox{\textwidth}{!}{\includegraphics{#1}}%
   \textsf{\caption[]{\footnotesize {#2}}}%
   \end{figure}}

%----- deal with pdf page-size stupidity
\special{papersize=8.5in,11in}
\setlength{\pdfpageheight}{\paperheight}
\setlength{\pdfpagewidth}{\paperwidth}

% no more bad lines!
\sloppy\sloppypar

% A better underline!
% tex.stackexchange.com/questions/36894/underline-omitting-the-descenders
\usepackage[T1]{fontenc}
\usepackage[latin1]{inputenc}
\usepackage{soul}
\usepackage{xparse}
\makeatletter

% http://www.guitex.org/home/it/forum/5-tex-e-latex/83195-la-libreria-hobby-tikz-non-funziona-piu#83203
\ExplSyntaxOn
\cs_if_exist:NF \prg_stepwise_function:nnnN { \cs_gset_eq:NN \prg_stepwise_function:nnnN \int_step_function:nnnN }
\cs_if_exist:NF \prg_stepwise_inline:nnnn { \cs_gset_eq:NN \prg_stepwise_inline:nnnn \int_step_inline:nnnn }
\ExplSyntaxOff

\ExplSyntaxOn
\cs_new:Npn \white_text:n #1
  {
    \fp_set:Nn \l_tmpa_fp {.01}
    \fp_mul:Nn \l_tmpa_fp {#1}
    \llap{\textcolor{white}{\the\SOUL@syllable}\hspace{\fp_to_decimal:N \l_tmpa_fp em}}
    \llap{\textcolor{white}{\the\SOUL@syllable}\hspace{-\fp_to_decimal:N \l_tmpa_fp em}}
  }
\NewDocumentCommand{\whiten}{ m }
    {
      \prg_stepwise_function:nnnN {1}{1}{#1} \white_text:n
      % \int_step_function:nnnN {1}{1}{#1} \white_text:n
    }
\ExplSyntaxOff

\NewDocumentCommand{ \varul }{ D<>{5} O{0.2ex} O{0.1ex} +m } {%
\begingroup
\setul{#2}{#3}%
\def\SOUL@uleverysyllable{%
   \setbox0=\hbox{\the\SOUL@syllable}%
   \ifdim\dp0>\z@
      \SOUL@ulunderline{\phantom{\the\SOUL@syllable}}%
      \whiten{#1}%
      \llap{%
        \the\SOUL@syllable
        \SOUL@setkern\SOUL@charkern
      }%
   \else
       \SOUL@ulunderline{%
         \the\SOUL@syllable
         \SOUL@setkern\SOUL@charkern
       }%
   \fi}%
    \ul{#4}%
\endgroup
}
\makeatother


% Underline paragraph titles
\usepackage[explicit]{titlesec}
\titleformat{\paragraph}[runin]
    {\normalfont\normalsize\bfseries}{\theparagraph}{1em}
    {\varul{#1}}

\pagestyle{myheadings}
\markright{\textsf{\footnotesize %
                   IMPRS summer school /
                   2016 /
                   MCMC exercises by Dan Foreman-Mackey}}

% Single-spacing.
\def\baselinestretch{1.1}

\include{figures/numbers-mh}
\include{figures/numbers-emcee}

\newcommand{\question}{\emph}

\begin{document}

In this exercise session you will implement a simple Metropolis MCMC sampler
in your programming language of choice.
You will test this sampler on a simple toy problem~--~drawing samples from a
two-dimensional Gaussian~--~and then on two more realistic problems.
The exercises are designed to take longer than the allotted 45~minutes so
don't worry if you don't finish.
Hopefully people with all levels of experience will learn something!
If you have previous experience with MCMC and if you have already implemented
a Metropolis sampler, why don't you take this time to implement a more
sophisticated method (ensemble, Hamiltonian, nested, or otherwise), try out a
new programming language, or help out other students with less experience.

\paragraph{Toy problem: a two-dimensional Gaussian}

In the following section you will implement your own Metropolis MCMC sampler
and test it by drawing samples from a two-dimensional Gaussian but the first
step is to implement the probability distribution as a function that takes in
a two-dimensional vector $\theta$ and returns:
\begin{eqnarray}
\ln p(\theta) &=& -\frac{1}{2}\,\theta^\mathrm{T}\,\Sigma^{-1}\,\theta
    + \mathrm{constant}
\end{eqnarray}
where
\begin{eqnarray}
\Sigma &=& \left(\begin{array}{cc}
1.0 & -0.08 \\
-0.08 & 0.01 \\
\end{array}\right)
\end{eqnarray}
Note: you definitely want to compute $\ln p$ (not just $p$).
\question{Why?}

\question{To test your probability function, ensure that the difference
between the function evaluated at two (random) points in parameter space is
the same as what I get.}
For example, make sure that you find the following:
\begin{eqnarray}
\ln p([0.5, -0.1]^\mathrm{T}) - \ln p([-0.01, 0.3]^\mathrm{T}) &\equiv&
    11.80847\ldots \quad.
\end{eqnarray}
If you don't get this result on the first try, debug your function until you
do.

\paragraph{A homegrown Metropolis sampler}

In this section, you will implement a Metropolis MCMC sampler to draw samples
from the distribution defined in the previous section.
As a reminder, the steps in a Metropolis MCMC are described here in words:
\begin{enumerate}

\item Initialize the parameters as $\theta(t=0)$.

\item Propose an update $q = \theta(t) + \sigma\,\delta$ where $\sigma$ is a
tuning parameter and $\delta$ is drawn from a zero mean, unit variance
Gaussian. Alternatively, you can try updating $\theta$ only one dimension.

\item Compute the acceptance probability $r =
\mathrm{min}\left(1,\,\frac{p(q)}{p(\theta(t))}\right)$.

\item Draw $u$ from a uniform distribution between 0 and 1. If $u < r$, accept
the proposal and set $\theta(t+1) = q$ to the chain. Otherwise, reject the
proposal and set $\theta(t+1) = \theta(t)$. Append $\theta(t+1)$ to the chain.
\emph{Note: in every iteration of this procedure a new entry is added to the
chain~--~even if the sample isn't accepted!}

\item Go to step 2 and repeat.

\end{enumerate}
Here are some suggestions to keep in mind for your implementation:
\begin{itemize}

\item In this function, you should track the acceptance fraction.
Keep track of how many proposals you accept.

\item As mentioned above, don't compute or use $p(\theta)$ directly~--~you
should only ever use the logarithm of this function.

\item Your implementation must support parameter \emph{vectors}.
For this problem, we're working in 2-D but don't special case to that because
it might be necessary to be able to sample more parameters later.

\item Similarly, you should implement the sampler in such a way that it is
easy to swap out a different probability function. Most programming languages
allow you to pass function pointers so taking the log-probability function as
an argument is probably the best interface.

\item Finally, to start the sampler, you must choose an initial location in
parameter space. Don't hard-code this location~--~take it as an
argument~--~because we'll test different initialization methods later.

\end{itemize}
\question{Implement the Metropolis MCMC method as a function with a calling
sequence something like}
\begin{center}
\texttt{samples, acc\_frac = metropolis(log\_prob\_func, sigma, inital\_theta,
num\_steps)}
\end{center}
This function should return the chain of samples and the accumulated
acceptance fraction.


\paragraph{Testing, tuning, and convergence (aka time to break your sampler)}

MCMC methods are notoriously hard to test rigorously so instead you can just
test your implementation qualitatively for today.
To start, initialize at a reasonable location (\question{What is a good way to
choose this in general?}) and \question{run some chains and look at a few
diagnostics}
\begin{itemize}

\item What is the acceptance fraction? Is this unreasonably high or low?
Note: this is a very easy problem so the acceptance fraction will
probably be higher than you would get in any problem in The Real
World\textsuperscript{\textsf{TM}}.

\item Compute the sample mean and covariance matrix of the chain. Is this
close to what you would expect? What do you expect and why? What happens as
you run the chain for longer?

\item Plot the trace (value as a function of step number) for each parameter.
See Figure~\ref{fig:traces} for an example of what to expect.

\item Plot the scatterplot matrix or corner plot to visualize the covariances
in the problem.
In Python you can use
\textsf{corner.py}\footnote{\url{http://corner.readthedocs.io}} and other
plotting libraries probably have similar functionality but, otherwise, you can
just make a scatterplot of the chain values.
Figure~\ref{fig:corner} shows an example using \textsf{corner.py}.

\item Try changing $\sigma$ and your initialization and remake these plots to
see what happens.
What happens as you make $\sigma$ extremely large $10^4$ or small $10^{-4}$?
What happens if you move the initial guess far away from the peak of the
distribution?

\end{itemize}

As discussed in lecture, the best quantification of sampler performance is the
integrated autocorrelation time $\tau_\mathrm{int}$.\footnote{Note: there isn't
just \emph{one} autocorrelation time.
$\tau_\mathrm{int}$ will, in general, be different for every different
$f(\theta)$ in Equation~\ref{eq:integral}.}
This provides an estimate of the number of steps of MCMC required to obtain an
independent sample.
Since MCMC is used to compute integrals, an estimate of $\tau_\mathrm{int}$ is
required to compute the sampling uncertainty on the approximation
\begin{eqnarray}\label{eq:integral}
\int f(\theta)\,p(\theta)\,\mathrm{d}\theta &\approx&
    \frac{1}{N} \sum_{n=1}^N f(\theta^{(n)}) \quad
    \mathrm{where}\,\theta^{(n)} \sim p(\theta)
\end{eqnarray}
and the sampler that produces a chain with a smallest value of
$\tau_\mathrm{int}$ at fixed computational cost is the best.

\question{Write or find a function that computes the autocorrelation function
of a one-dimensional chain and compute this for a few choices of $f(\theta)$.}
Figure~\ref{fig:autocorr} gives a few suggestions for functions to consider.
\question{Then, find or implement a function that estimates the integrated
autocorrelation time}:
\begin{eqnarray}
\tau_\mathrm{int,est} &\approx& C(0) + 2\,\sum_{\tau=1}^W C(\tau)
\end{eqnarray}
where $C(\tau)$ is the autocorrelation function and $W$ is a tuning parameter.
Note: autocorrelation time estimates are always very noisy and you need to run
a long chain to get a reliable estimate.
Furthermore, the choice of $W$ can be subtle. Try several values of $W$ or
read A.~Sokal's notes\footnote{See pages 15--16 of this note:
\url{http://www.stat.unc.edu/faculty/cji/Sokal.pdf}} on how to choose $W$ more
robustly.
If you are working in Python, \textsf{emcee}\footnote{Versions 2.2.x and
greater: \url{http://dan.iel.fm/emcee/current/api/\#autocorrelation-analysis}.}
includes an implementation of this method.

For $\sigma = 0.1$, I find integrated autocorrelation times of \taua~steps and
\taub~steps for $f(\theta)=\theta_1$ and $f(\theta)=\theta_2$ respectively.
\question{Do you find the same? If not, why not?
Try changing $\sigma$ to see how the autocorrelation function and times
change.
Can you find a better choice of $\sigma$? What if you use different values of
$\sigma$ for each parameter? How many tuning parameters are there in this
method?}

\ssfigure{figures/traces.pdf}{0.5}{%
These are traces.
\label{fig:traces}}

\ssfigure{figures/corner.pdf}{0.5}{%
And a scatterplot matrix.
\label{fig:corner}}

\ssfigure{figures/autocorr.pdf}{0.5}{%
The autocorrelation function.
\label{fig:autocorr}}

\ssfigure{figures/ensemble.pdf}{0.5}{%
The autocorrelation function.
\label{fig:ensemble}}

% \begin{multicols}{2}
% {\centering\bf REFERENCES\par}
% \vspace{0.2em}
% \begin{thebibliography}{}%
% \raggedright\raggedbottom\scriptsize\setlength{\parskip}{-0.5em}%


% \end{thebibliography}
% \end{multicols}

\end{document}
